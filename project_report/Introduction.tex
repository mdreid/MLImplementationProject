Classification is the problem of identifying the category to which observations belong, given that the identity of the category is unknown at the beginning. A classification algorithm required to place these observations or instances into groups based on information of each of the object’s attribute. A great deal of algorithms have been proposed in the last decades for classifying data in a variety of domains.

	Decision tree learning is one of the most widely used and practical methods for inductive inference. [1] --- putting some shortcomings of decision tree here => think of adding k-d tree to improve. ---

	The final decision tree nodes contain univariate splits as regular decision trees, but the leaf nodes contain with nearest neighbor search with KD trees, which is a variant of standard Nearest Neighbor classifier. Our results on a handwritten digit dataset shows that this hybrid approach has clear advantages when compared against its counterpart, Decision Tree.

