\section{Related Work}
Quinlan proposed C4.5 as an improvement to ID3 (cite original C4.5 work). Quinlan further improved C4.5 by changing how the algorithm handles continuous features (cite - improved use of cts attributes). This is relevant to our work because we treated pixel values as continuous numbers in the range of 0 to 255 (inclusive). 

The MNIST database of scanned images of handwritten digits is a well-known and frequently used dataset (cite - LeCun et al., 1998). It contains 60,000 examples for trainining and 10,000 examples for testing. Various machine learning algorithms have been used on this data, which makes it useful for benchmarking against. The algorithm that performs best on the dataset, as of the time of this writing, is a committee of 35 convolutional neural networks (cite Ciresan et al.)

Much work has looked at ensemble methods, in which multiple learning algorithms are used to make predictions. We will now provide a brief review of previous work that has used k-NN, decision trees, or both in an ensemble method.
