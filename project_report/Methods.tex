\section{Methods}

\subsection{Dataset}
	Dataset that we used is MNIST database of handwritten digits. This dataset has pixel values (integers from 0 to 255) for 60,000 training examples. The MNIST database contains a separate testing set of 10,000 examples.
	\subsection{Algorithm}
	%\begin{algorithm}[H]
		\begin{flushleft}
			\item{MakeSubTree(set of training instances $D$)}:
		\end{flushleft}
		\begin{algorithmic}[1]

			\renewcommand{\algorithmicrequire}{\textbf{Input: }}
			\renewcommand{\algorithmicensure}{\textbf{Output:}}
			\REQUIRE set of  training instances $D$
			\ENSURE  decision tree

			\STATE{$C \leftarrow DetermineCandidateSplits(D)$}
			\IF {stopping criteria met}
			\STATE make a leaf node
			\STATE store instances D here // (possibility 1)
			\ELSE
			\STATE make an internal node $N$
			\STATE{$S \leftarrow FindBestSplit(D,C)$}
			\ENDIF

			\FOR {each outcome $k$ of $S$}
			\STATE {$D_{k} \leftarrow$} {subset of instances that have outcome $k$}
			\STATE {$k^{th}$ child of $N$}{$ \leftarrow MakeSubtree(D_{k})$}
			\ENDFOR
			\RETURN {subtree rooted at $N$}
		\end{algorithmic}
	%\end{algorithm}
	\subsection{Dimensionality Reduction}
