\section{Methods}
\subsection{Combining Approaches}
In our hybrid approach, we combine the decision tree algorithm with a Nearest Neighbor algorithm (we found that $k$-d tree Nearest Neighbor implementations performed better than exhaustive search). Figure \ref{fig:workflow} shows the control flow of operations on the data set.

After learning and pruning a decision tree using the C4.5 algorithm, for each leaf of the decision tree we associate a $k$-d tree.

\subsection{Dimensionality Reduction}
Since the number of features in MNIST dataset is relatively large (784 features), we initially experienced prohibitively long training times. In order to reduce the training time, we use scikit-learn Python machine learning API to apply principal components analysis (PCA) to the training set to reduce the number of features to 100. Note that we generate the transform object using training and tuning data and then apply it to training, tuning, and testing data.  This way we avoid any overfitting issues that might result from the inclusion of our testing data.

Next, this reduced-dimensionality training data is used to train a hybrid decision tree. After the pruning step, and at each of the leaf nodes, we build a Nearest Neighbor model to make predictions instead of using the normal majority-approach. The following sections will give more details about our hybrid classifier's implementation.

\subsection{Decision Tree}
After running PCA, we had a set of 100 real-valued features. 
Therefore, we decided on an algorithm implementation
We implemented decision trees using Quinlan's C4.5 algorithm (cite - Quinlan).
At a high level, we grew a complete tree and then greedily pruned it.
To choose splits, we choose the split with the maximum information gain.
We prune the tree using a tuning set 10\% of the size of the training set not including these tuning examples.
For each pair of leaf nodes, we compare the accuracy of the tree on the tuning set with the pair to the accuracy of the tree on the tuning set without the pair.
If the latter is less than or equal to the former, we prune the leaves.
We repeat this pruning process recursively.
We provide data on the the size of the tree before and after pruning in the Results section.


\subsection{Nearest Neighbor}
$k$-\textit{NEAREST NEIGHBOR} ($k$-NN) is an instance-based, or ``lazy'' learning method that stores the training examples in the learning phase, and estimates the target function for each new instance to be classified. This algorithm assumes all instances correspond to points in the $n$-dimensional space, and to classify an instance $i$, $k$-NN finds the instance's $k$ closest neighbors and assigns \^{c} as $i$'s label, where \^{c} is the most common label among those $k$ nearest neighbors. The nearest neighbors can be found using one of these distance methods: Euclidean, Squared Euclidean, Hamming, or Manhattan. In this paper, we use Squared Euclidean as our distance method for $k$-NN.
\subsubsection{Exhaustive Search}
Exhaustive Search is a straightforward way to implement $k$-NN. That is, we store all the training instances in the learning step. To classify an instance, we loop through all the training instances to find its $k$ nearest neighbors. Hence, the time complexity to find an instance's nearest neighbor uisng Exhaustive Search is $O(N)$. For this reason, the prediction time can be very long if the training set is large. This is why we also implement $k$-d tree in the next section to find an instance nearest neighbor.

\subsubsection{$k$-d Tree}
The $k$-d tree version of $k$-NN stores the training instances in a $k$-d tree data structure during the training step, and then uses this tree to find the nearest neighbor to an instance during the predicting step. The time complexity to find an instance's nearest neighbor can be greatly improved by $k$-d Tree to an average of $O(log_2N)$. Our implementation for $k$-d tree is based on this wiki page \cite{KDWiki}.