\section{Methods}
	\subsection{Algorithms and Combining Methods}
		In our hybrid approach, we combine the decision tree algorithm with a nearest neighbor algorithm (k-d tree or exhaustive search). Figure \ref{fig:workflow} shows the flowchart of the approach over the MNIST dataset. 
		
		First of all, since the number of features in MNIST dataset is too big (784 features), to reduce the training time, we apply principal component analysis (PCA) to the training set to reduce the number of features to 100. Then the training data is input to the decision tree algorithm to obtain a classification tree. After that the tree is pruned, and at each of the leaf nodes, we build a nearest neighbor model to make predictions instead of using the normal majority-approach. The following sections will give more details about Decision Tree and Nearest Neighbor.
		
	\subsection{Decision Tree}
    After running PCA, we had a set of 100 real-valued features. 
Therefore, we decided on an algorithm implementation
We implemented decision trees using Quinlan's C4.5 algorithm (cite - Quinlan).
At a high level, we grew a complete tree and then greedily pruned it.
To choose splits, we choose the split with the maximum information gain.
We prune the tree using a tuning set 10\% of the size of the training set not including these tuning examples.
For each pair of leaf nodes, we compare the accuracy of the tree on the tuning set with the pair to the accuracy of the tree on the tuning set without the pair.
If the latter is less than or equal to the former, we prune the leaves.
We repeat this pruning process recursively.
We provide data on the the size of the tree before and after pruning in the Results section.


	\subsection{Nearest Neighbor}
		\subsubsection{K-D Tree}
		\subsubsection{Exhaustive Search}

